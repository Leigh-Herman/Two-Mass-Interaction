\documentclass[12pt]{article}
\usepackage{amsmath}
\usepackage{amsfonts}
\usepackage{amssymb}
\usepackage{hyperref}

\title{Numerical Relativity Project: Geodesics Around a Massive Body}
\author{Your Name}
\date{\today}

\begin{document}

\maketitle

\tableofcontents

\section*{Introduction}

In this project, we simulate the interaction between a small mass orbiting a large mass using numerical relativity techniques. The goal is to visualize the geodesics in the spacetime field around the large mass and analyze the interaction between the two masses. The project is divided into the following steps:
\begin{enumerate}
    \item Define the problem framework.
    \item Set up the geodesic equations.
    \item Implement numerical methods for solving the equations.
    \item Visualize the results to interpret the dynamics.
    \item Represent the spacetime field to complement the geodesic visualization.
\end{enumerate}

Each step is detailed in its respective chapter, including Python code and mathematical derivations.

\newpage

\section{Defining the Problem Framework}
In this chapter, we introduce the Schwarzschild metric as the spacetime model and explain its mathematical formulation. We derive the Christoffel symbols, which are key to describing geodesics in this spacetime.

\newpage

\section{Setting Up the Geodesic Equations}
This chapter formulates the geodesic equations as a system of ordinary differential equations (ODEs). We explain the mathematical derivations and the conversion of the second-order equations to a first-order system.

\newpage

\section{Numerical Integration of Geodesics}
We discuss the implementation of numerical methods such as the Runge-Kutta method to solve the geodesic equations. The Python code for integration is presented and explained in detail.

\newpage

\section{Visualization of Geodesics}
This chapter focuses on visualizing the geodesics in spacetime. We explore methods for creating 2D and 3D plots of the trajectories and overlaying them on representations of the spacetime curvature.

\newpage

\section{Representation of the Spacetime Field}
Finally, we calculate and visualize the Ricci curvature tensor and other relevant spacetime quantities to provide additional insight into the dynamics.

\newpage

\section*{Conclusion}
This project demonstrates how numerical techniques can be applied to simulate and visualize the interaction of masses in a relativistic framework. Future work can explore more complex spacetimes, such as those around rotating or charged masses.

\end{document}
